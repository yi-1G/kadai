%\documentstyle[epsf,twocolumn]{jarticle}       %LaTeX2.09仕様
\documentclass[twocolumn]{jsarticle}     %pLaTeX2e仕様
%%%%%%%%%%%%%%%%%%%%%%%%%%%%%%%%%%%%%%%%%%%%%%%%%%%%%%%%%%%%%%
%%
%%  基本 バージョン
%%
%%%%%%%%%%%%%%%%%%%%%%%%%%%%%%%%%%%%%%%%%%%%%%%%%%%%%%%%%%%%%%%%
\setlength{\topmargin}{-45pt}
%\setlength{\oddsidemargin}{0cm} 
\setlength{\oddsidemargin}{-7.5mm}
%\setlength{\evensidemargin}{0cm} 
\setlength{\textheight}{24.1cm}
%setlength{\textheight}{25cm} 
\setlength{\textwidth}{17.4cm}
%\setlength{\textwidth}{172mm} 
\setlength{\columnsep}{11mm}

\kanjiskip=.07zw plus.5pt minus.5pt


%【節がかわるごとに(1.1)(1.2) …(2.1)(2.2)と数式番号をつけるとき】
%\makeatletter
%\renewcommand{\theequation}{%
%\thesection.\arabic{equation}} %\@addtoreset{equation}{section}
%\makeatother

%\renewcommand{\arraystretch}{0.95} 行間の設定

%%%%%%%%%%%%%%%%%%%%%%%%%%%%%%%%%%%%%%%%%%%%%%%%%%%%%%%%
\usepackage[dvipdfm]{graphicx}   %pLaTeX2e仕様(要\documentstyle ->\documentclass)
\usepackage{cite}
\usepackage{url}
%%%%%%%%%%%%%%%%%%%%%%%%%%%%%%%%%%%%%%%%%%%%%%%%%%%%%%%%

\begin{document}

\twocolumn[
\noindent
\hspace{1em}

\today
\hfill
\ \ 1191201005 池田祐介

\vspace{2mm}
\hrule
\begin{center}
{\Large \bf 12/7〜12/13 報告書}
\end{center}
\hrule
\vspace{3mm}
]


\section{進捗}
\begin{itemize}
\item 夏目漱石の小説『吾輩は猫である』の文章(neko.txt)をデータとして例として使用した
\item データに対して形態素解析することで形容詞を含む文を検出し、その形容詞部分を[MASK]トークンに変換して、文頭に[CLE]トークン、文末に[SEP]トークンを付加した文をデータセットとして用いた
\item 東北大学の研究室が公開している事前学習済みBERTモデルを用いて[MASK]部分の推定を行い、上位100件の候補の中から形容詞のみを最大5個抽出して表示した
\item 抽出した候補の中に[MASK]に変換される前の形容詞と一致するものがあれば1、なければ0として、1を出力する割合を出すと0.366であった 
\end{itemize}

\section{次に取り組むこと}
\begin{itemize}
\item 元となる文に対して係り受け解析をする
\item 形容詞以外の形容語(形容動詞、副詞など)についても同様の処理を行ってみる
\item 直喩法(〜のような)についても同様の処理を行ってみる
\item もう少し大きなデータセットを用いる
\end{itemize}

\nocite{*}
\bibliography{index}
\bibliographystyle{junsrt} %参考文献出力スタイル

\end{document}
